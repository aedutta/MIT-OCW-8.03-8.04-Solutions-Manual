\begin{sol}
\begin{enumerate}[label=\textbf{(\alph*)}]
\item 
Adding a phase shift to the bottom leg is equivalent to effectively multiplying the state by the matrix:
$$\text{PS} = \begin{pmatrix}
1 & 0 \\
0 & e^{i\delta}
\end{pmatrix}$$
The output is represented by:
\begin{align*}
\text{output} &=\text{(BS2)(PS)(BS1)}\binom{\alpha}{\beta} \\
&=
\frac{1}{\sqrt{2}}
\begin{pmatrix}
1 & 1 \\
1 & -1
\end{pmatrix}
\begin{pmatrix}
1 & 0 \\
0 & e^{i\delta}
\end{pmatrix}
\frac{1}{\sqrt{2}}
\begin{pmatrix}
-1 & 1 \\
1 & 1
\end{pmatrix}
\binom{\alpha}{\beta} \\
&= \frac{1}{2}
\begin{pmatrix}
1 & e^{i\delta} \\
1 & -e^{i \delta}
\end{pmatrix}
\begin{pmatrix}
-1 & 1 \\
1 & 1
\end{pmatrix}
\binom{\alpha}{\beta} \\
&= \frac{1}{2}
\begin{pmatrix}
-1+e^{i\delta} & 1+e^{i\delta}\\
-1-e^{i\delta} & 1-e^{i\delta}
\end{pmatrix}
\binom{\alpha}{\beta} \\
&= \frac{1}{2}
\binom{(-1+e^{i\delta})\alpha+(1+e^{i\delta})\beta}{-(1+e^{i\delta})\alpha+(1-e^{i\delta})\beta}
\end{align*}
The squares of the norms give the respective probabilities:
\begin{align*}
    P_0 &= \frac{1+(2\beta^2-1)\cos\delta}{2} \\
    P_1 &= \frac{1-(2\beta^2-1)\cos\delta}{2} \\
\end{align*}
\item Due to symmetry, this is equivalent to simply switching $\beta$ and $\alpha$, or effectively flipping the setup so that $P_0$ and $P_1$ swap and $\beta$ and $\alpha$ swap. Therefore, the new probabilities are:
\begin{align*}
    P_0 &= \frac{1-(1-2\beta^2)\cos\delta}{2} \\
    P_1 &= \frac{1+(1-2\beta^2)\cos\delta}{2} \\
\end{align*}
\item By placing a phase shift at the top and bottom, it is equivalent to multiplying by
$$\text{PS} = \begin{pmatrix}
e^{i\delta} & 0 \\
0 & e^{i\delta}
\end{pmatrix} = e^{i\delta} \begin{pmatrix}
1 & 0 \\
0 & 1
\end{pmatrix}
$$
Therefore, all this does is shift the final phase of both by rotating the state by an angle $\delta$, which will not change the overall probability. This will lead to the result derived in the lecture:
\begin{align*}
    P_0 &= \beta^2 \\
    P_1 &= \alpha^2 \\
\end{align*}
\end{enumerate}
\end{sol}