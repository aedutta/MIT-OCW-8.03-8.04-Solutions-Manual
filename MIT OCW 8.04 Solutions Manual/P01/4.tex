\begin{sol}
\begin{enumerate}[label=\textbf{(\alph*)}] 
\item We use the Taylor expansion of $e^{i\theta}$.
\begin{align*}
    e^{i\theta} &= 1 + i\theta + \frac{1}{2!}(i\theta)^2 + \frac{1}{3!}(i\theta)^3 + \frac{1}{4!}(i\theta)^4 + \frac{1}{5!}(i\theta)^5 + \dots\\
    &= 1 + i\theta - \frac{1}{2!}\theta^2 - \frac{1}{3!}i\theta^3 + \frac{1}{4!}\theta^4 + \frac{1}{5!}i\theta^5 + \dots
\end{align*}
Separating these functions into even and odd powers gives us 
\begin{align*}
    e^{i\theta} &= \left(1 - \frac{1}{2!}\theta^2 + \frac{1}{4!}\theta^4 + \dots\right) + i\left(\theta -\frac{1}{3!}\theta^3 + \frac{1}{5!}\theta^5 + \dots\right)\\
    e^{i\theta} &= \cos\theta + i\sin\theta.
\end{align*}
\item $a$ is the real part of the complex exponential, while $b$ is the imaginary part. Using Euler's formula: 
\begin{align*}
    a = r\cos\theta \\
    b = r\sin\theta
\end{align*}
Then, note that the magnitude of the complex exponential is just the value of $r$. Additionally, the angle $\theta$ can be considered using the ratio of imaginary to real components of the complex exponential.
\begin{align*}
r = \sqrt{a^2 + b^2} \\
\theta = \arctan{\frac{b}{a}}
\end{align*}
\item
\begin{enumerate}[label=\textbf{(\roman*)}]
    \item Note that $i = e^{\frac{i\pi}{2}}$, since $\sin{\frac{\pi}{2}} = 1$ and $\cos{\frac{\pi}{2}} = 0$. Thus, using the properties of exponents:
\begin{align*}
    re^{i\theta} \cdot i &= e^{i\theta} \cdot e^{\frac{i\pi}{2}} \\
    &= re^{i(\theta + \frac{\pi}{2})}
\end{align*}
    \item $iz = i(a + ib) = ia - b$. Thus, the real part of $iz$ is $-b$.
\end{enumerate}
\item
\begin{enumerate}[label=\textbf{(\roman*)}]
\item 0, since its complex conjugate and the negative of its complex conjugate are both equal to itself, 0.
    \item The complex conjugate flips the sign of the imaginary term, so taking the complex conjugate twice leaves us with $(z^*)^* = z$. For the second part of the question, note that $re^{-i\theta} = r\cos{-\theta} + i\sin{-\theta}$. Since cosine is an odd function and sine is an odd function, $re^{-i\theta} = r\cos\theta - i\sin\theta = a -bi$, which is the conjugate of the original complex number.
    \item The real part is the same and the imaginary part becomes negative, by the definition of the complex conjugate.
    \item $zz^* = (a + bi)(a - bi) = a^2 + b^2$. Using complex exponentials, $zz^* = re^{i\theta} * re^{-i\theta} = r^2$. Since $r = |z|$, $zz^* = |z|^2$.
\end{enumerate}
\item First, consider the double angle identities. We can represent $\cos{2\theta}$ and $\sin{2\theta}$ by the real and imaginary parts of $e^{2i\theta} = e^{i\theta} * e^{i\theta}$. Expanding:
\begin{align*}
    e^{2i\theta} &= e^{i\theta} \cdot e^{i\theta} \\ &= (\cos\theta + i\sin\theta)(\cos\theta + i\sin\theta) \\ &= \cos^2\theta + 2i\sin\theta\cos\theta - \sin^2\theta \\ &= (\cos^2\theta - \sin^2\theta) + i(2\sin\theta\cos\theta).
\end{align*}
Taking the real and imaginary parts of this expression, we obtain the double angle identities:
\begin{align*}
    \cos{2\theta} = \cos^2\theta - \sin^2\theta \\
    \sin{2\theta} = 2\sin\theta\cos\theta
\end{align*}
The same thing can be done to solve for the triple angle identities, by finding the real and imaginary components of $e^{3i\theta} = e^{i\theta} * e^{i\theta} * e^{i\theta}$
\begin{align*}
    e^{3i\theta} &= e^{i\theta} \cdot e^{i\theta} \cdot e^{i\theta} \\ &= (\cos\theta + i\sin\theta)(\cos\theta + i\sin\theta)(\cos\theta + i\sin\theta) \\ &= \cos^3\theta + 3i\cos^2\theta\sin\theta - 3\cos\theta\sin^2\theta - i\sin^3\theta \\ &= (4\cos^3\theta - 3\cos\theta) + i(3\sin\theta - 4\sin^3\theta)
\end{align*}
Taking the real and imaginary components, we obtain the triple angle identities:
\begin{align*}
    \cos{3\theta} = 4\cos^3\theta - 3\cos\theta \\
    \sin{3\theta} = 3\sin\theta - 4\sin^3\theta
\end{align*}
Now, the more general case of such identities can be found, with two arbitrary angles $A$ and $B$ added with each other in a similar fashion.
\begin{align*}
    e^{i(A + B)} &= e^{iA} \cdot e^{iB} \\ &= (\cos{A} + i\sin{A})(\cos{B} + i\sin{B}) \\ &= \cos{A}\cos{B} + i\sin{A}\cos{B} + i\sin{B}\cos{A} - \sin{A}\sin{B} \\ &= (\cos{A}\cos{B} - \sin{A}\sin{B}) + i(\sin{A}\cos{B} + \sin{B}\cos{A})
\end{align*}
Taking the real and imaginary components again, we obtain the sum and difference identities:
\begin{align*}
    \cos{(A + B)} = \cos{A}\cos{B} - \sin{A}\sin{B} \\
    \sin{(A + B)} = \sin{A}\cos{B} + \sin{B}\cos{A}
\end{align*}
\end{enumerate}
\end{sol}