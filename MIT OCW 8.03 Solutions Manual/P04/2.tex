\begin{sol}
\begin{enumerate}[label=\textbf{(\alph*)}] 
\item Since the matrices commute, the modes will be simultaneous eigenvectors of the symmetry transformation and the interaction. There are then two modes. These two solution are $x(t)$ and $\tilde{x}(t) = \mathcal{S}x(t)$.

\item When we have a symmetry matrix either we don't undergo any transformation on the eigenvalue $\lambda = 1$, or it will undergo a transformation on the eigenvalue $\lambda = -1$. \footnote{
This is justified because $\det (A - \lambda I) = \lambda^2 - 1 = 0\implies \lambda = 1, \lambda = -1$.}
Since the 1s of the symmetry matrix are off the diagonal, when we multiply $\mathcal{S}$ by the components of the eigenvector we will switch the eigenvalues. For example:
\[
\begin{pmatrix}
0 & -1 \\
-1 & 0
\end{pmatrix}
\begin{pmatrix}
A \\
B
\end{pmatrix}
=
\begin{pmatrix}
-B \\
A
\end{pmatrix}
\]
We need the absolute value of $A$ and $B$ to be the same if we want it to only change by a factor of 1 or -1 upon reflection. Therefore, if we try something such as 
\[
\begin{pmatrix}
A \\
A
\end{pmatrix}
\]
and multiply by the symmetry matrix, we get 
\[
\begin{pmatrix}
0 & -1 \\
-1 & 0
\end{pmatrix}
\begin{pmatrix}
A \\
A
\end{pmatrix}
=
\begin{pmatrix}
-A \\
-A
\end{pmatrix}
\]
The ratio of the amplitudes of the components of the eigenvector are then 1.
\end{enumerate}
\end{sol}