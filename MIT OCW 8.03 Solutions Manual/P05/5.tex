\begin{sol}
\begin{enumerate}[label=\textbf{(\alph*)}] 
\item Let us analyze the forces involved. There is a vertical drag force $b\ddot{y}$, a tension directed downwards at an angle $\theta$ to the horizontal, and a horizontal normal force that acts on the pole in the opposite direction of the tension. 
\vspace{3mm}

\noindent Applying Newton's law in the vertical direction gives us 
\[m\ddot{y} = -T\sin\theta + F_d.\]
Writing the force in differential form and assuming small angles yields,
\[m\ddot{y} = -T\frac{\partial y}{\partial x} - b\frac{\partial y}{\partial t}.\]
The mass of the ring is negligible and thus 
\[-T\frac{\partial y}{\partial x} - b \frac{\partial y}{\partial t} = 0 \implies \boxed{\frac{\partial y}{\partial x} = -\frac{b}{T}\frac{\partial y}{\partial t}}.\]

\item The wave can be written as a superposition of the incident and reflected pulse. In other words, 
\[y (x,t) = f(vt - x) + g(vt + x).\]
The reflected wave $g(x + vt)$ is unknown, but we can use boundary conditions used in part (a) to solve for it. 
\begin{align*}
    \pdv{y}{x} &= f' (vt - x) + g' (vt + x) \\
    \pdv{y}{t} &= v (f' ( vt - x) + g' (vt + x))
\end{align*}
Defining are coordinate system such that the hoop is at $x = 0$, we then can then write 
\[\pdv{y}{x} = -\frac{b}{T}\pdv{y}{T}\implies f' (vt) + g' (vt) = -\frac{bv}{T} (f' ( vt) + g' (vt)).\]
Isolating $g(vt)$ tells us that 
\[g'(vt) = \frac{bv - T}{bv + T}f' (vt).\]
To now solve for $g$ in terms of $f$ we must integrate this expression. Substituting $u = vt$ gives us 
\[\int g' (u) du = \int \frac{bv - T}{bv + T}f' (u) du \implies g (vt) = \boxed{\frac{bv - T}{bv + T} f(vt)}.\]
\end{enumerate}
\end{sol}